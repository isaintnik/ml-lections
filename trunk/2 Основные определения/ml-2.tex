\documentclass[14pt, fleqn, xcolor={dvipsnames, table}]{beamer}
\usepackage[T2A]{fontenc}
\usepackage[utf8]{inputenc}
\usepackage[english,russian]{babel}
\usepackage{amssymb,amsfonts,amsmath,mathtext}
\usepackage{cite,enumerate,float,indentfirst}

\usepackage{tikz}                   
\usetikzlibrary{shadows}

% \usepackage{enumitem}
% \setitemize{label=\usebeamerfont*{itemize item}%
%   \usebeamercolor[fg]{itemize item}
%   \usebeamertemplate{itemize item}}

\graphicspath{{images/}}

\usetheme{Madrid}
\usecolortheme{seahorse}

\setbeamercolor{footline}{fg=Blue!50}
\setbeamertemplate{footline}{
  \leavevmode%
  \hbox{%
  \begin{beamercolorbox}[wd=.333333\paperwidth,ht=2.25ex,dp=1ex,center]{}%
    И. Кураленок, Н. Поваров, Яндекс
  \end{beamercolorbox}%
  \begin{beamercolorbox}[wd=.333333\paperwidth,ht=2.25ex,dp=1ex,center]{}%
    Санкт-Петербург, 2013
  \end{beamercolorbox}%
  \begin{beamercolorbox}[wd=.333333\paperwidth,ht=2.25ex,dp=1ex,right]{}%
  Стр. \insertframenumber{} из \inserttotalframenumber \hspace*{2ex}
  \end{beamercolorbox}}%
  \vskip0pt%
}
\newcommand\indentdisplays[1]{%
     \everydisplay{\addtolength\displayindent{#1}%
     \addtolength\displaywidth{-#1}}}
\newcommand{\itemi}{\item[\checkmark]}

\title{Машинное обучение: начало\\\small{}}
\author[]{\small{%
И.~Куралёнок,
Н.~Поваров}}
\date{}

\begin{document}

\begin{frame}
\maketitle
\small
\begin{center}
\vspace{-60pt}
\normalsize {\color{red}Я}ндекс \\
\vspace{80pt}
\footnotesize СПб, 2013
\end{center}
\end{frame}

\section{Постановка задачи}
\begin{frame}
\frametitle{Задача на сегодня}
\emph{Задача:} отделить ``хороших'' студентов от ``плохих'' \\
\\
\emph{Формально:} предсказать средний балл следующей сессии
\end{frame}

\begin{frame}
\frametitle{План работы}
\begin{enumerate}
\item Определиться с тем кто такой студент
\item Как из этого определения можно понять хороший он или нет
\item Выработать чувство прекрасного
\item Решить полученную задачу оптимизации
\item Оценить качество полученного решения
\item Проанализировать результаты
\end{enumerate}
\end{frame}

\section{Векторизация и факторы}
\begin{frame}
\frametitle{Векторизация}
{\tinyВ данном случае не про графику.} \\
\\
\emph{\textbf{Векторизация:} перевод представления о предмете в векторное выражение.}\\
\\
\emph{Компоненты полученного в результате векторизации вектора будем называть \textbf{факторами}}
\end{frame}

\section{Векторизация и факторы}
\begin{frame}
\frametitle{Факторы про студента}
\begin{itemize}
% инф, инт
  \item Баллы ЕГЭ по математике при поступлении
  \item Процент пропусков лекций
  \item Количество друзей = $\|$ ВК $\cup$ О $\cup$ FB $\|$
% инф, инт, шум
  \item Литры пива в неделю
  \item Текущий средний балл по мнению родителей
% инф, не инт
  \item Пол
  \item Тип школы
% неинф, инт
  \item Расстояние от дома до универа (км)
  \item Диаметр головы
% неинф, неинт
  \item Предпочитаемый ряд в аудитории
  \item Наличие планшета
\end{itemize}
\end{frame}
\begin{frame}[t]\frametitle{Множество на котором обучаемся $L$}
\small
\begin{tabular}{l||l|l|l|l|l|l|l|l|l|l|l||c}
\hline
Имя & 1 & 2 & 3 & 4 & 5 & 6 & 7 & 8 & 9 & 10 & 11 & Итог \\
\hline\hline
Вася & 87 & 2 & 0.3 & 145 & 4.7 & 1 & 3 & 12 & 53.7 & 3 & 1 & 3.8 \\
\hline
\end{tabular}
\end{frame}

\section{Решающая и целевая функции}
\begin{frame}[t]\frametitle{Решающая функция}
    $$\begin{array}{c}
      h_{w_0}(x) = w_0^{T}x \\
      \\
      w_0, x \in \mathbb{R}^n \\
      \end{array}
    $$
\begin{itemize}
  \item Простая
  \item Универсальная
  \item Легко интерпретируемая
\end{itemize}
\end{frame}

\begin{frame}[t]\frametitle{Целевая функция}
    $$
      w_0 = \arg\min_w \sum_{(x_i,y_i) \in L} \| h_w(x_i) - y_i \|
    $$
\begin{itemize}
  \item MSE --- годное первое приближение в $\mathbb{R}$
  \item Интерпретируемое значение $\min$
\end{itemize}
\end{frame}

\begin{frame}[t]\frametitle{Решение и его интерпретация}
    $$
      \begin{array}{ll}
      w_0 &= \arg\min_w \sum_{(x_i,y_i) \in L} \left\| x_i^Tw - y_i \right\| \\
       &= \arg\min_w \left\|X^Tw - y\right\|^2 \\
      &= \left(XX^T\right)^{-1}Xy \\
      \end{array}
    $$
\end{frame}

\section{Оценка качества обучения}
\begin{frame}[t]\frametitle{Хорош ли полученный результат}
\begin{itemize}
  \item Отличается ли от КО
  \item Сколько можно выжать из данных
  \item Можно ли верить его компонентам
  \item Воспроизводится ли результат
  \item Насколько можно доверять предсказанию
\end{itemize}
\end{frame}

\subsection{Качество решения}
\begin{frame}[t]\frametitle{По невязке}
$$
t_L = \sum_{(x_i,y_i) \in L} \left\| h_{w_0}(x_i) - y_i \right\|
$$
\begin{itemize}
  \item Просто посчитать
  \item Рассказывает о работе на тренировочном множестве
  \item Ничего не говорит о качестве предсказания
\end{itemize}
\end{frame}

\subsection{Качество предсказания}

\begin{frame}[t]\frametitle{По невязке на другом множестве I}
Можно поделить множество на 2 части, настроить на одной, проверить на другой
$$
  \begin{array}{ll}
    DS = L \cup T, L \cap T = \o \\
    \\
    w_0 = \arg\min_w \sum_{(x_i,y_i) \in L} \left\| x_i^Tw - y_i \right\| \\
    t_T = \sum_{(x_i,y_i) \in T} \left\| h_{w_0}(x_i) - y_i \right\|
  \end{array}
$$
\end{frame}

\begin{frame}[t]\frametitle{По невязке на другом множестве II}
\begin{itemize}
  \item Расскажет о качестве предсказания с точностью до деления на $L$ и $T$
  \item Использует меньше данных в обучении
  \item Если исходное множество не показательно, то деление нас не спасет
  \item Можно посмотреть на $t_L$ и $t_T$
\end{itemize}
\end{frame}

\subsection{Стабильность решения}

\begin{frame}[t]\frametitle{По стабильности решения}
Поделим несколько раз и посмотрим на то как меняется $w_0$.
\begin{itemize}
  \item Стабильные компоненты заслуживают веры
  \item Если все нестабильно --- беда-беда
\end{itemize}
\end{frame}

\section{Анализ данных}
\begin{frame}[t]\frametitle{Можно ли сделать лучше?}
\begin{itemize}
  \item Мало данных или много факторов?
  \begin{itemize}
    \item Все ли факторы одинаково хороши?
    \item Может их можно скомбинировать?
    \item Стоит ли одинаково верить всем факторам?
  \end{itemize}
  \item Может быть в данных что-то нечисто?
  \begin{itemize}
    \item Все ли мы можем объяснить?
    \item А набирали данные правильно?
    \item Не подсматриваем ли мы в ответ?
    % строчки данных, от которых некоторые фичи стреляют
    % данные в лерн содержат результаты про тест, например средние значения по хосту, при делении по запросам
    \item Все ли важные примеры представленны в данных?
  \end{itemize}
\end{itemize}
\end{frame}

\subsection{Не все факторы одинаково полезны}
\begin{frame}[t]\frametitle{Все ли факторы одинаково полезны}
\begin{itemize}
  \item Можем ли мы обойтись без какого-нибудь фактора?
  \item А если фактор преобразовать, может его станет проще использовать?
  \item Если есть похожие факторы, наверное это можно учесть.
  \item Стот ли рассмотреть комбинации нескольких факторов?
  \item Что мы делаем, если фактор посчитать нельзя?
\end{itemize}
\end{frame}

\subsubsection{Информативность фактора}
\begin{frame}[t]\frametitle{Полезна ли информация, которую несет фактор?}

\begin{itemize}
  \item Что мы делаем, если фактор посчитать нельзя?
\end{itemize}
\end{frame}

\subsubsection{Интерпретируемость в рамках модели}
\begin{frame}[t]\frametitle{Как сделать фактор ``съедобным'' для модели?}
Мы использовали линейную модель, в которой, например, категориальные факторы не могут быть эффективно задействованы. Что же делать?
\begin{itemize}
  \item Что мы делаем, если фактор посчитать нельзя?
\end{itemize}
\end{frame}
\subsubsection{Шумность/смещенность}
\subsection{Взаимодействие факторов}
\subsubsection{Корреляция}
\subsubsection{Interaction}
\subsection{Выбросы}
\subsection{Приспособляемость}
\section{Подбор целевой функции}
\subsection{А ту ли задачу мы решаем?}
\subsection{Логистическая регрессия}
\section{Заключение}
\begin{frame}{Информативные факторы}
Интерпретируемые
\begin{itemize}
% инф, инт
  \item Баллы ЕГЭ по математике при поступлении
  \item Литры пива в неделю
  \item Процент пропусков лекций
  \item Количество друзей = $\|$ ВК $\cup$ О $\cup$ FB $\|$
\end{itemize}
% инф, инт, шум  
Шумный
\begin{itemize}
  \item Текущий средний балл по мнению родителей
\end{itemize}
% инф, не инт
Неинтерпретируемые
\begin{itemize}
  \item Пол
  \item Тип школы
\end{itemize}
\end{frame}

\begin{frame}[t]{Неинформативные}
Интерпретируемые
\begin{itemize}
  \item Расстояние от дома до универа
  \item Диаметр головы
\end{itemize}
Неинтерпретируемые
\begin{itemize}
  \item Предпочитаемый ряд в аудитории
  \item Наличие планшета
\end{itemize}    
\end{frame}

\end{document} 
