\documentclass[14pt, fleqn, xcolor={dvipsnames, table}]{beamer}
\usepackage[T2A]{fontenc}
\usepackage[utf8]{inputenc}
\usepackage[english,russian]{babel}
\usepackage{amssymb,amsfonts,amsmath,mathtext}
\usepackage{cite,enumerate,float,indentfirst}
\usepackage{cancel}

\usepackage{tikz}                   
\usetikzlibrary{shadows}

% \usepackage{enumitem}
% \setitemize{label=\usebeamerfont*{itemize item}%
%   \usebeamercolor[fg]{itemize item}
%   \usebeamertemplate{itemize item}}

\graphicspath{{images/}}

\usetheme{Madrid}
\usecolortheme{seahorse}
\renewcommand{\CancelColor}{\color{red}}

\setbeamercolor{footline}{fg=Blue!50}
\setbeamertemplate{footline}{
  \leavevmode%
  \hbox{%
  \begin{beamercolorbox}[wd=.333333\paperwidth,ht=2.25ex,dp=1ex,center]{}%
    И. Кураленок, Н. Поваров, Яндекс
  \end{beamercolorbox}%
  \begin{beamercolorbox}[wd=.333333\paperwidth,ht=2.25ex,dp=1ex,center]{}%
    Санкт-Петербург, 2013
  \end{beamercolorbox}%
  \begin{beamercolorbox}[wd=.333333\paperwidth,ht=2.25ex,dp=1ex,right]{}%
  Стр. \insertframenumber{} из \inserttotalframenumber \hspace*{2ex}
  \end{beamercolorbox}}%
  \vskip0pt%
}
\newcommand\indentdisplays[1]{%
     \everydisplay{\addtolength\displayindent{#1}%
     \addtolength\displaywidth{-#1}}}
\newcommand{\itemi}{\item[\checkmark]}

\title{Генетические алгоритмы\\\small{}}
\author[]{\small{%
И.~Куралёнок,
Н.~Поваров}}
\date{}

\begin{document}

\begin{frame}
\maketitle
\small
\begin{center}
\vspace{-60pt}
\normalsize {\color{red}Я}ндекс \\
\vspace{80pt}
\footnotesize СПб, 2013
\end{center}
\end{frame}

\section{Линейные модели}
\section{Линейная регрессия}
\subsection{Статистические свойства решения}
\subsection{Теорема Гаусса-Маркова}
\subsection{Расширение на несколько целевых функций}
\section{Линейная классификация}
\subsection{Простая линейная модель и ее вариации}
\subsection{Линейный дискриминантный анализ (LDA)}
\subsection{Логистическая регрессия}
\section{Бонус: метрики классификации}
\section{Домашнее задание}


\end{document}
