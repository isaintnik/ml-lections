\documentclass[14pt, fleqn, xcolor={dvipsnames, table}]{beamer}
\usepackage[T2A]{fontenc}
\usepackage[utf8]{inputenc}
\usepackage[english,russian]{babel}
\usepackage{amssymb,amsfonts,amsmath,mathtext}
\usepackage{cite,enumerate,float,indentfirst}
\usepackage{cancel}
\usepackage{graphicx}
\usepackage{animate}
\usepackage{ulem}

\usepackage{tikz}
% \usepackage{enumitem}
\usetikzlibrary{shadows}

% \usepackage{enumitem}
% \setitemize{label=\usebeamerfont*{itemize item}%
%   \usebeamercolor[fg]{itemize item}
%   \usebeamertemplate{itemize item}}

\graphicspath{{images/}}

\usetheme{Madrid}
\usecolortheme{seahorse}
\renewcommand{\CancelColor}{\color{red}}

\setbeamercolor{footline}{fg=Blue!50}
\setbeamertemplate{footline}{
  \leavevmode%
  \hbox{%
  \begin{beamercolorbox}[wd=.333333\paperwidth,ht=2.25ex,dp=1ex,center]{}%
    И. Кураленок, Н. Поваров, Яндекс
  \end{beamercolorbox}%
  \begin{beamercolorbox}[wd=.333333\paperwidth,ht=2.25ex,dp=1ex,center]{}%
    Санкт-Петербург, 2014
  \end{beamercolorbox}%
  \begin{beamercolorbox}[wd=.333333\paperwidth,ht=2.25ex,dp=1ex,right]{}%
  Стр. \insertframenumber{} из \inserttotalframenumber \hspace*{2ex}
  \end{beamercolorbox}}%
  \vskip0pt%
}
\newcommand\indentdisplays[1]{%
     \everydisplay{\addtolength\displayindent{#1}%
     \addtolength\displaywidth{-#1}}}
\newcommand{\itemi}{\item[\checkmark]}

\newenvironment{mydescription}[1]
  {\begin{list}{}%  
   {\renewcommand\makelabel[1]{\color{blue}##1:\hfill}%
   \settowidth\labelwidth{\makelabel{#1}}%
   \setlength\leftmargin{\labelwidth}
   \addtolength\leftmargin{\labelsep}}}
  {\end{list}}

\title{GBDT. Смешанные модели. \\\small{Немного о работающих алгоритмах}}
\author[]{\small{%
И.~Куралёнок,
Н.~Поваров}}
\date{}
\begin{document}

\begin{frame}
\maketitle
\small
\begin{center}
\vspace{-60pt}
\normalsize {\color{red}Я}ндекс \\
\vspace{80pt}
\footnotesize СПб, 2014
\end{center}
\end{frame}

\section{Gradient Bossted Descision Trees}
\begin{frame}{От CART к лесам}
\end{frame}

\begin{frame}{Случай произвольной целевой функции}
\end{frame}

\begin{frame}{VC оценка для GBDT}
\end{frame}

\begin{frame}{Стабильность работы GBDT}{в зависимости от целевой функции}
\end{frame}

\begin{frame}{Распределение трудоемкости в GBDT}
\end{frame}

\section{Смешанные модели}

\begin{frame}{BagBoo}
\end{frame}

\begin{frame}{BooBag}
\end{frame}

\begin{frame}{Где это добро применять?}
\end{frame}

\section{Деревья, которые работают}

\begin{frame}{Еще раз о информации в дереве}
\end{frame}

\begin{frame}{Деревянная регуляризация}
\end{frame}

\begin{frame}{Исправление дисперсии}
\end{frame}

\begin{frame}{Я.ЗабывчивыеДеревья от Cliff Brunk}
\end{frame}

\section{Парные производные}

\begin{frame}{Парные производные от Андрея Гулина}{или как устроен -F/-U}
\end{frame}

\section{Заключение}

\begin{frame}{Что мы сегодня узнали}
\begin{itemize}
  \item Как раскладывать целевую функцию в ряд деревьев
  \item Какие характеристики важны в бустинге
  \item Как играть в гольф из лука
  \item Можно ли сделать еще более работающие деревья
  \item Подробности решающей функции можно использовать для еще более точного разложения
\end{itemize}
\end{frame}

\end{document}
