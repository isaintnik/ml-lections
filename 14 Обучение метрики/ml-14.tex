\documentclass[14pt, fleqn, xcolor={dvipsnames, table}]{beamer}
\usepackage[T2A]{fontenc}
\usepackage[utf8]{inputenc}
\usepackage[english,russian]{babel}
\usepackage{amssymb,amsfonts,amsmath,mathtext}
\usepackage{cite,enumerate,float,indentfirst}
\usepackage{cancel}
\usepackage{graphicx}
\usepackage{animate}

\usepackage{tikz}
% \usepackage{enumitem}
\usetikzlibrary{shadows}

% \usepackage{enumitem}
% \setitemize{label=\usebeamerfont*{itemize item}%
%   \usebeamercolor[fg]{itemize item}
%   \usebeamertemplate{itemize item}}

\graphicspath{{images/}}

\usetheme{Madrid}
\usecolortheme{seahorse}
\renewcommand{\CancelColor}{\color{red}}

\setbeamercolor{footline}{fg=Blue!50}
\setbeamertemplate{footline}{
  \leavevmode%
  \hbox{%
  \begin{beamercolorbox}[wd=.333333\paperwidth,ht=2.25ex,dp=1ex,center]{}%
    И. Кураленок, Н. Поваров, Яндекс
  \end{beamercolorbox}%
  \begin{beamercolorbox}[wd=.333333\paperwidth,ht=2.25ex,dp=1ex,center]{}%
    Санкт-Петербург, 2013
  \end{beamercolorbox}%
  \begin{beamercolorbox}[wd=.333333\paperwidth,ht=2.25ex,dp=1ex,right]{}%
  Стр. \insertframenumber{} из \inserttotalframenumber \hspace*{2ex}
  \end{beamercolorbox}}%
  \vskip0pt%
}
\newcommand\indentdisplays[1]{%
     \everydisplay{\addtolength\displayindent{#1}%
     \addtolength\displaywidth{-#1}}}
\newcommand{\itemi}{\item[\checkmark]}

\newenvironment{mydescription}[1]
  {\begin{list}{}%
   {\renewcommand\makelabel[1]{\color{blue}##1:\hfill}%
   \settowidth\labelwidth{\makelabel{#1}}%
   \setlength\leftmargin{\labelwidth}
   \addtolength\leftmargin{\labelsep}}}
  {\end{list}}

\title{Обучение метрики\\\small{по Tutorial on Metric Learning by Brian Kulis}}
\author[]{\small{%
И.~Куралёнок,
Н.~Поваров}}
\date{}
\begin{document}

\begin{frame}
\maketitle
\small
\begin{center}
\vspace{-60pt}
\normalsize {\color{red}Я}ндекс \\
\vspace{80pt}
\footnotesize СПб, 2013
\end{center}
\end{frame}

\section{Содержание}
\section{Постановка проблемы}
\begin{frame}{Метрики в контексте IBL}
В прошлый раз использовали несколько buzzword, сегодня поговорим об одном из них --- дистанции между точками
\end{frame}

\begin{frame}{Пример с Михалычем}
\end{frame}

\begin{frame}{Картинки с линейной разделимостью и тем что с ними хотим сделать}
\end{frame}

\begin{frame}{Mahalanobis distance}
как пример того, что можно улучшить пространство линейным преобразованием
\end{frame}

\begin{frame}{Постановка задачи обучения метрики}
Как задачи обучения с учителем
\end{frame}


\section{Обучение линейного преобразования}

\begin{frame}{Варианты целевой функции}

\end{frame}

\begin{frame}{Варианты регуляризации}

\end{frame}

\begin{frame}{Варианты условий}

\end{frame}

\begin{frame}{MMC}
\end{frame}


\begin{frame}{MMC: алгоритм}
\end{frame}


\begin{frame}{Schultz and Joachims}
\end{frame}

\begin{frame}{Schultz and Joachims: Алгоритм}
\end{frame}

\begin{frame}{NCA}
Как пример сильно другого таргета
\end{frame}

\begin{frame}{Онлайн постановка, POLA}
\end{frame}

\begin{frame}{POLA: алгоритм}
\end{frame}


\begin{frame}{ITML как пример LogDet}
\end{frame}

\begin{frame}{ITML алгоритм}
\end{frame}

\begin{frame}{Не-Махаланобисовы методы: LDF}
Local distance functions setup
\end{frame}

\section{Кернелизация}
\begin{frame}{Проблема линейной неразделимости}
Где-то мы это уже видели, сделаем kernel
\end{frame}

\begin{frame}{Кернелизация ITML}
\end{frame}

\begin{frame}{Кернелизация ITML: алгоритм}
\end{frame}

\begin{frame}{Постановка kernel learning для metric learning}
\end{frame}

\section{Заключение}
\begin{frame}{Что мы сегодня узнали}
\begin{itemize}
  \item Расстояние можно менять
  \item Удобно работать с неметриками, убрав правило треугольника
  \item Точно как поставить задачу обучения метрики
  \item Есть несколько способов обучить линейное преобразование, в том числе online
  \item Сообщество любителей ядер постаралось и на этой ниве, что позволяет решать линейно неразделимые задачи
\end{itemize}
\end{frame}
\end{document}
