\documentclass[14pt, fleqn, xcolor={dvipsnames, table}]{beamer}
\usepackage[T2A]{fontenc}
\usepackage[utf8]{inputenc}
\usepackage[english,russian]{babel}
\usepackage{amssymb,amsfonts,amsmath,mathtext}
\usepackage{cite,enumerate,float,indentfirst}

\usepackage{tikz}                   
\usetikzlibrary{shadows}

% \usepackage{enumitem}
% \setitemize{label=\usebeamerfont*{itemize item}%
%   \usebeamercolor[fg]{itemize item}
%   \usebeamertemplate{itemize item}}

\graphicspath{{images/}}

\usetheme{Madrid}
\usecolortheme{seahorse}

\setbeamercolor{footline}{fg=Blue!50}
\setbeamertemplate{footline}{
  \leavevmode%
  \hbox{%
  \begin{beamercolorbox}[wd=.333333\paperwidth,ht=2.25ex,dp=1ex,center]{}%
    И. Кураленок, Н. Поваров, Яндекс
  \end{beamercolorbox}%
  \begin{beamercolorbox}[wd=.333333\paperwidth,ht=2.25ex,dp=1ex,center]{}%
    Санкт-Петербург, 2013
  \end{beamercolorbox}%
  \begin{beamercolorbox}[wd=.333333\paperwidth,ht=2.25ex,dp=1ex,right]{}%
  Стр. \insertframenumber{} из \inserttotalframenumber \hspace*{2ex}
  \end{beamercolorbox}}%
  \vskip0pt%
}
\newcommand\indentdisplays[1]{%
     \everydisplay{\addtolength\displayindent{#1}%
     \addtolength\displaywidth{-#1}}}
\newcommand{\itemi}{\item[\checkmark]}

\title{Машинное обучение: оценка методов обучения с учителем\\\small{}}
\author[]{\small{%
И.~Куралёнок,
Н.~Поваров}}
\date{}

\begin{document}

\begin{frame}
\maketitle
\small
\begin{center}
\vspace{-60pt}
\normalsize {\color{red}Я}ндекс \\
\vspace{80pt}
\footnotesize СПб, 2013
\end{center}
\end{frame}

\section{Постановка задачи и классификация способов оценки}
\begin{frame}{Задача на сегодня}
\emph{Задача:} Есть метод обучения и данные, на которых обучаемся. Хотим понять хорошо ли будет работать решающая функция на практике.\\

\flushleft{\em ``If you can't measure it, you can't improve it''} \\
\flushright{\textbf{—-- Lord Kelvin}} \\

\flushleft{\em ``Гораздо легче что-то измерить, чем понять, что именно вы измеряете.''} \\
\flushright{\textbf{—-- Джон Уильям Салливан}}
\end{frame}

\subsection{Источники проблемы}
\begin{frame}{Источник проблемы}
$$
F_0 = \arg \max_{F(D)} \mu_{\xi \sim U\left(\Gamma\right)}T(y_{\xi}, F(x_{\xi}))
$$
Мы обучаемся на одном множестве, а работаем на другом. А что, если эти множества отличаются?
\end{frame}

\begin{frame}{Свойства выборки}
\flushleft{\em ``Иными словами, репрезентативная выборка представляет собой микрокосм, меньшую по размеру, но точную модель генеральной совокупности, которую она должна отражать.''}\\
\flushright{\textbf{--- Дж. Б. Мангейм, Р. К. Рич}}\\
\flushleft
Такого сложно достичь, поэтому хотим лишь ``несмещенности'' по параметрам обучения:
\small
$$\begin{array}{rl}
F_0 =& \arg \max_{F} \mu_{\xi \sim U(D)} T(y_{\xi}, F(x_{\xi})) \\
    =& \arg \max_{F} \mu_{\xi \sim U(\Gamma)} T(y_{\xi}, F(x_{\xi}))
\end{array}$$
\end{frame}

\begin{frame}{Способы повлиять на несмещенность}
Способы понятны:
\begin{itemize}
  \item Поменять класс решающих функций
  \item Поменять выборку
\end{itemize}
Но для любых изменений необходимы измерения, и чем тоньше изменения, тем более точный инструмент измерения нужен.
\end{frame}

\subsection{Классификация способов оценки}
\begin{frame}{Способы повлиять на несмещенность}
Способы понятны:
\begin{itemize}
  \item Поменять класс решающих функций
  \item Игры с точностью обучения
  \item Поменять выборку
\end{itemize}
Но для любых изменений необходимы измерения, и чем тоньше изменения, тем более точный инструмент измерения нужен.
\end{frame}

\begin{frame}{Известные способы оценки}
Оценка по принципу ``чёрного ящика'':
\begin{itemize}
  \item Оценка в боевых условиях (на пользователях)
  \item Кросс-валидация
\end{itemize}
Оценка по принципу ``прозрачного ящика'':
\begin{itemize}
  \item VС оценки
  \item PAC-Bayes bounds
  \item Оценки по Воронцову
\end{itemize}
\end{frame}

\section{Black box оценка}
\begin{frame}{Оценка в боевых условиях}

\end{frame}

\subsection{Cross-validation}

\begin{frame}{Cross-fold validation}
Можно организовать разными способами:
\begin{itemize}
  \item 2-fold
  \item k-fold
  \item Random sub-sampling (e.g. bootstrapping) 
  \item Leave-one-out (LOOCV)
\end{itemize}
\end{frame}
\section{Сложность модели}
\subsection{Пример с полиномами}

\begin{frame}{Сложность модели}
\end{frame}

\begin{frame}{Семейство полиномов $p$-й степени}
\end{frame}

\subsection{VC-dimension}
\begin{frame}{Размерность Вапника-Червоненкиса}
\end{frame}

\section{Виды ошибок}
\begin{frame}{Overfit vs. underfit}
\end{frame}

\begin{frame}{Как это выглядит в полиномах (underfit)}
\end{frame}

\begin{frame}{Как это выглядит в полиномах (fit)}
\end{frame}

\begin{frame}{Как это выглядит в полиномах (overfit)}
\end{frame}

\begin{frame}{Как это выглядит в пространстве решений}
\end{frame}

\begin{frame}{Что в какой ситуации делать}
\end{frame}

\begin{frame}{Понимаем в какой ситуации находимся}
\end{frame}

\section{Glass box оценка}
\begin{frame}{Теоретическая оценка}
Цели оценки:
\begin{itemize}
  \item Можно ли понять какой метод круче по объему данных
  \item Предсказать сложность на которой достигается fit
\end{itemize}
\end{frame}

\subsection{VC-оценка и ее применение}
\begin{frame}{Оценка по методу Вапника-Червоненкиса}
Общая модель оценки.
Ошибка как функция vc-размерности.
\end{frame}

\subsection{Оценка в слабой аксиоматике Воронцова}
\begin{frame}{Оценка в слабой аксиоматике Воронцова}
\end{frame}

\section{Про неточные решения}
\begin{frame}{Соображения об информации в параметрах}
\end{frame}

\begin{frame}{Игры с шагом}
\end{frame}

\section{Overfit on validate}
\begin{frame}{Как еще можно переобучиться?}
\end{frame}

\begin{frame}{Классическое трехчастное деление данных}
\end{frame}

\end{document} 
